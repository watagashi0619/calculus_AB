\section{$\mathbb{R}^n$上の関数と連続性}
\begin{itemize}
    \item $f:\mathbb{R}^n\to\mathbb{R}^m$と書いた時,$f$は$\mathbb{R}^n$から$\mathbb{R}^m$への関数という。
    \item $A\subset\mathbb{R}^n$に対して$f$が定義され,$f(x)$の値は$B\subset\mathbb{R}^m$に入る時,$f:A\to B$と書く。
    \item $A$を$f$の定義域(domain)という。
    \item $f(A):=\{f(x)\in\mathbb{R}^m:x\in A\}$を値域(range)という。
    \item $c\subset\mathbb{R}^m$に対して$f^{-1}(c):=\{x\in A:f(x)\in c\}$
    \begin{note}
        $f^{-1}$は逆写像を用いて定めていない。逆関数の存在もいえない。
    \end{note}
    \item $f:A\to B$が単射である,もしくは1対1である$\overset{def}{\Leftrightarrow}$$[x,y\in A,x\neq y \Rightarrow f(x)\neq f(y)]$
    \item $f:A\to B$が全射である,もしくは上への写像である$\overset{def}{\Leftrightarrow}$$[\forall z\in B,\exists x\in A$\ s.t.\ $f(x)=z]$
    \item $A\subset\mathbb{R}^n$とする。\\
    $f:A\to\mathbb{R}^m:$単射\ に対し,逆関数$f^{-1}$は$f(A)\to\mathbb{R}^n$が$z\in f(A)$に対して$f(x)=z$となるただ1つの点$x\in A$を$f^{-1}(z)$と定めることで定義される。\\
    $f(x)=z$となるただ1つの点を定められないと仮定,すなわち$\exists x'\in A$でも$f(x')=z$とすると,$x\neq x'$のとき,$f$は単射であるので$f(x)\neq f(x')$。このとき$z=f(x)\neq f(x')=z$となり矛盾する。
    \begin{note}
        単射かつ全射,すなわち全単射のときは$f$は必ず逆関数をもつ。\\(i.e.$f:A\to B:$全単射$\Rightarrow$$\exists f^{-1}:B\to A$)
    \end{note}
    \item 関数の成分表示\\
    $f:A\to\mathbb{R}^n$に対し
    \[
    f(x)=(f^1(x),f^2(x),\cdots,f^m(x))
    \]
    と書くことで$m$個の成分表示
    \[
    f^1,f^2,\cdots,f^m:A\to\mathbb{R}
    \]
    が決まる。
    \item 関数の極限\\
    $f:A\to\mathbb{R}^m\ (A\subset\mathbb{R}^n)$に対して
    \[
    \lim_{x\to a}f(x)=b \overset{def}{\Leftrightarrow} \forall\epsilon>0,\exists\delta>0\ s.t.\ |x-a|<\delta(x\in A)\Rightarrow|f(x)-b|<\epsilon
    \]
    \item 連続性
    \begin{itemize}
        \item $f:A\to\mathbb{R}^m$が$a\in A$で連続である$\overset{def}{\Leftrightarrow}$$\displaystyle\lim_{x\to a}f(x)=f(a)$
        \item $f$が$A$上連続である$\overset{def}{\Leftrightarrow}$$\forall a\in A$で$f$が連続である
    \end{itemize}
\end{itemize}

\newpage

\begin{framed}
    \begin{thm}
        $f:A\to\mathbb{R}^m\ (A\subset\mathbb{R}^n)$が連続$\Leftrightarrow$$\forall U\subset \mathbb{R}^m:$openに対して$\exists V\subset\mathbb{R}^n:$open\ s.t.\ $f^{-1}(U)=V\cap A$\footnotemark
    \end{thm}
\end{framed}

\footnotetext{
$f:$連続$\Leftrightarrow$open setの逆像がopen set。\\
ある写像が連続であることとその映った先の開集合の引き戻しが開集合であることは同値である。
}

\begin{proof} \\
    ($\Rightarrow$について)\\
    $f:$連続とする。$a\in f^{-1}(U)$ならば$f(a)\in U$。$U$はopenより,$\exists B_a:$openb lock(開方体) s.t.\ $f(a)\in B_a\subset U$とできる。$f:$連続ゆえ$a$を含む十分小さいopen block $C_a$をとると,$x\in C_a\cap A$ならば$f(x)\in B_a$となる(これにより$a\in f^{-1}(U)\to C_a$が定まった)。$\displaystyle V:=\bigcup_{a\in f^{-1}(U)}C_a=\{x\in\mathbb{R}^n:\exists a\in f^{-1}(U)$\ s.t.\ $x\in C_a\}$とすると,$V:$openで$f^{-1}(U)=V\cap A$となる。
    \footnote{
    $f^{-1}(U)\subset V\cap A$は定義より明らか。$V\cap A\subset f^{-1}(U)$について,$\forall x\in V\cap A$に対して$\exists a\in f^{-1}(U)$\ s.t.\ $x\in C_a\cap A$。このとき$f(x)\in B_a\subset U$より$x\in f^{-1}(U)$。
    }
    \\
    ($\Leftarrow$について)\\
    $\forall\epsilon>0$に対して$U:=\{y\in\mathbb{R}^m:|y-f(a)| <\epsilon\}$とする。このとき$U$はopen setなので$\exists V\subset\mathbb{R}^n:$open set s.t. $f^{-1}(U)=V\cap A$とできる。$V$は$\mathbb{R}^n$のopen setであり,$a\in V\cap A$であるから$\exists\delta>0$ s.t. $\{x\in A:|x-a|<\delta\}\subset V\cap A$。ゆえに,$f(\{x\in A:|x-a|<\delta\})\subset U$となり\footnote{
    $\therefore\{x\in A:|x-a|<\delta\}\subset V\cap A=f^{-1}(U)$
    },
    $f$は$A$で連続である。
    \footnote{
    $f(\{x\in A:|x-a|<\delta\})\subset\{y\in\mathbb{R}^m:|y-f(a)|<\epsilon\}$より,$\forall\epsilon>0,\exists\delta>0$ s.t. $x\in A,|x-a|<\delta\Rightarrow|f(x)-f(a)|<\epsilon$
    }
\end{proof}

\begin{framed}
    \begin{thm}\label{th1.9}
        $f:A\to\mathbb{R}^m(A\subset\mathbb{R}^n)$が連続で$A$がcompactならば$f(A)\subset\mathbb{R}^m$はcompactである。
    \end{thm}
\end{framed}

\begin{proof}
    $\mathscr{O}$を$f(A)$の開被覆とする。$\mathscr{O}$中の各開集合$U$に対し$\mathbb{R}^n$の開集合$V_U$で$f^{-1}(U)=V_U\cap A$となるものが存在する。$\mathscr{O}':=\{V_U\subset\mathbb{R}^n:U\in\mathscr{O}\}$とすると$\mathscr{O}'$は$A$の開被覆。$A$はcompactより$\mathscr{O}'$中の有限個$V_{U_1},V_{U_2},\cdots,V_{U_k}$がすでに覆っている,よって$f(A)$は$U_1,U_2,\cdots,U_k$で覆われる。
\end{proof}