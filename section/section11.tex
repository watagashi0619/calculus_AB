\section{重積分}
\begin{dfn*}[上積分・下積分]
	$A:\mathbb{R}^n$の閉方体,$f:A\to\mathbb{R}:$有界関数\\
このとき$\displaystyle\sup_{\Delta}s(f,\Delta)$と$\displaystyle\inf_{\Delta}S(f,\Delta)$は必ず存在する。これらを$\displaystyle\underline{\int_A} f$,$\displaystyle\overline{\int_A}f$で表し,前者を$f$の$A$上の下積分,後者を$f$の$A$上の上積分という。
\end{dfn*}
\begin{framed}
	\begin{thm}[重積分]\label{th3.10}
			$A\subset\mathbb{R}^n,B\subset\mathbb{R}^m:$閉方体,$f:A\times B\to\mathbb{R}:$可積分関数とする。$x\in A$に対し,$g_x:B\to\mathbb{R}$を$g_x(y):=f(x,y)$で定め,
			\[
				L(x):= \underline{\int_B}g_x= \underline{\int_B}f(x,y) dy
			\]
			\[
				U(x):= \overline{\int_B}g_x= \overline{\int_B}f(x,y) dy
			\]
			とおくと,関数$L(x),U(x)$はともに$A$上可積分で
			\[
			\begin{split}
			\int_{A\times B}f &= \int_A L\\
			&= \int_A\left(\underline{\int_B}f(x,y)dy\right)dx
			\end{split}
			\]
			\[
			\begin{split}
			\int_{A\times B}f &= \int_A U\\
			&= \int_A\left(\overline{\int_B}f(x,y)dy\right)dx
			\end{split}
			\]
			がともに成立する。この右辺の積分を重積分,あるいは累次積分という。
	\end{thm}
\end{framed}

\begin{proof}
	$\Delta_A,\Delta_B:A,B$の分割とすると,$\Delta=(\Delta_A,\Delta_B)$は$A\times B$の分割で$\Delta$の小方体は$C_A\times C_B$($=C$とする)($C_A,C_B$は$\Delta_A,\Delta_B$の小方体)の形となる。よって
	\[
	\begin{split}
		s(f,\Delta)
		&=\sum_C m_C(f)|C|\\
		&=\sum_{C_A}\sum_{C_B}m_{C_A\times C_B}(f)|C_A\times C_B|\\
		&=\sum_{C_A}\left(\sum_{C_B}m_{C_A\times C_B}(f)|C_B|\right)|C_A|
	\end{split}
	\]
	$\forall x\in C_A$を固定する。$\inf$の計算から
	\[
	m_{C_A\times C_B}(f)\leq m_{C_B}(g_x)\footnotemark
	\]
	\footnotetext{$C_A\times C_B$と$\{x\}\times C_B$}
	となるから
	\[
	\begin{split}
		\sum{C_B}m_{C_A\times C_B}(f)|C_B|
		&\leq \sum_{C_B}m_{C_B}(g_x)|C_B|\\
		&\leq \underline{\int_B} g_x\\
		&=L(x)
	\end{split}
	\]
	が成り立つ。したがって,
	\[
	\begin{split}
		\sum_{C_B}m_{C_A\times C_B}(f)|C_B|
		&\leq m_{C_A}(L)\\
		&=\inf\{L(x):x\in C_A\}
	\end{split}
	\]
	よって
	\[
	\begin{split}
	\sum_{C_A}\left(\sum_{C_B}m_{C_A\times C_B}(f)|C_B|\right)|C_A|
		&\leq\sum_{C_A} m_{C_A}(L)|C_A|\\
		&= s(L,\Delta_A)
	\end{split}
	\]
	ゆえに
	\begin{equation}\tag{1}
		s(f,\Delta)\leq s(L,\Delta_A)
	\end{equation}
	同様にして
	\begin{equation}\tag{2}
		S(U,\Delta_A)\leq S(f,\Delta)
	\end{equation}
	以上から
	\[
		s(f,\Delta)\underset{(1)}{\leq} s(L,\Delta_A)\leq S(L,\Delta_A) \underset{\footnotemark}{\leq} S(U,\Delta_A) \underset{(2)}{\leq} S(f,\Delta)
	\]
	\footnotetext{$L(x)\leq U(x)\Rightarrow S(L,\Delta_A)\leq S(U,\Delta_A) $,注釈なしの不等号の前後で極限をとって考える。}
	$f$は可積分だから
	\[
	\sup_{\Delta_A}s(L,\Delta_A)=\inf_{\Delta}S(f,\Delta)=\int_{A\times B}f
	\]
	が成立しているので
	\[
		\sup_{\Delta_A}s(L,\Delta_A)=\inf_{\Delta_A}S(L,\Delta_A)=\int_{A\times B}f
	\]
	が成立する。よって$L$は$A$上可積分であり,
	\[
	\int_{A\times B}f=\int_A L
	\]
	である。$U$に対しては同様にして次の不等式
	\[
		s(f,\Delta)\leq s(L,\Delta_A)\leq S(U,\Delta_A)\leq S(U,\Delta_A)\leq S(f,\Delta)
	\]
	を用いればよい。
\end{proof}

\newpage

\begin{note}\
	\begin{itemize}
			\item 同様に,累次積分の順序を逆にした公式
			\[
			\int_{A\times B}f=\int_B\left(\underline{\int_A}f(x,y)dx\right)dy=\int_B\left(\overline{\int_A}f(x,y)dx\right)dy
			\]
			が成り立つ。
			\item 各$g_x(y)=f(x,y)$が$\forall x\in A$で$y$に関して可積分な場合は
			\[
			\int_{A\times B}f=\int_A\left(\int_B f(x,y)dy\right)dx
			\]
			が成り立つ。特に$f$が連続ならば成立する。
			\item $g_x$が可積分でない$x\in A$が高々有限個の場合,この有限個以外の$x$に対して$\displaystyle L(x)=\int_B f(x,y)dy$であり,関数$L(x)$の積分は有限個の点で関数の値を変えても変わらないので、結局この場合も
			\[
			\int_{A\times B}f=\int_A\left(\int_B f(x,y)dy\right)dx
			\]
			が成り立つ。ただし,上記の有限個の$x$に対しては$\displaystyle \int_B f(x,y)dy$は,たとえば$0$と決めていることとする。
			\item $g_x$が可積分でない$x\in A$が有限個でない場合,定理\ref{th3.10}の形で使う必要がある。
			\begin{example}
				$f:[0,1]\times [0,1]\to\mathbb{R}$を
				\[
				f(x,y)=
				\begin{cases}
					1 & (x\in\mathbb{R}-\mathbb{Q})\\
					1 & (x\in\mathbb{Q},y\in\mathbb{R}-\mathbb{Q})\\
					1-\frac{1}{q} & (x,y\in\mathbb{Q},x=\frac{p}{q}:既約分数)
				\end{cases}
				\]
				と定めると、$f$は可積分(演習問題で示した)で,$\displaystyle \int_{[0,1]\times[0,1]}f=1$。一方で$x$が無理数なら$\displaystyle \int_0^1 f(x,y)dy=1$。$x$が有理数なら$g_x$はすべての点で不連続なので可積分でない。よって,$\displaystyle h(x)=\int_0^1 f(x,y)dy$が存在しない時には$0$とおくと$h(x)$は可積分にならない。\footnote{この場合,$1$とおけばたまたまうまくいくが,一般論として考えれば定義に戻るのが賢明。}
			\end{example}
			\item $A=[a_1,b_1]\times\cdots\times[a_n,b_n]\subset\mathbb{R}^n$のとき,$f:A\to\mathbb{R}$が例えば連続関数ならば,定理\ref{th3.10}を繰り返し用いて
			\[
			\int_A f = \int_{a_n}^{b_n}\left(\cdots\left(\int_{a_1}^{b_1}f(x^1,\cdots,x^n)dx^1\right)\cdots\right)dx^n
			\]
	\end{itemize}
\end{note}
