\section{陰関数}
\begin{example}
	$f(x,y)=x^2+y^2=1$で定まる$f:\mathbb{R}^2\to\mathbb{R}$を考える。$f(a,b)=0$となる点$(a,b)$を1つとり,$a\in A$,$b\in B$とする。このとき,$x\in A$に対して$f(x,y)=0$となる$y\in B$が唯一つ存在する。

	したがって$g:A\to\mathbb{R}\ (x(\in A)\mapsto y(\in\mathbb{R}))$が定まり,$f(x,g(x))=0$が成立する。このとき$g$は$C^1$級になっている。

	このような$g(x)$を$f(x,y)=0$で定まる陰関数という。
\end{example}

これを一般化して次の問題を考える。

\begin{qes}
	$f_i:\mathbb{R}^n\times\mathbb{R}^m\to\mathbb{R}\ (i=1,2,\cdots,m)$に対して,	$f_i(a^1,\cdots,a^n,b^1,\cdots,b^m)=0\ (i=1,\cdots,m)$となる点$(a,b)\in\mathbb{R}^n\times\mathbb{R}^m$を考える。$a=(a^1,\cdots,a^n)$の近傍の点$(x^1,\cdots,x^n)$に対し,$b=(b^1,\cdots,b^m)$の近傍の$(y^1,\cdots,y^m)$で
	\[
	f_i(x^1,\cdots,x^n,y^1,\cdots,y^m)=0\ (i=1,\cdots,m)
	\]
	となるものが唯一存在するか?
\end{qes}

\begin{framed}
	\begin{thm}[陰関数の定理]
		$f:\mathbb{R}^n\times\mathbb{R}^m\to\mathbb{R}^m$が点$(a,b)\in\mathbb{R}^n\times\mathbb{R}^m$を含むある開集合で,$C^1$級かつ$f(a,b)=0$とする。$m\times m$行列$M:=(D_{n+j}f^i(a,b))_{ij}\ (1\leq i,j\leq m)$と定め,$\det A\neq 0$とする。このとき,$\exists A:a$を含む開集合,$\exists B:b$を含む開集合 s.t. $\forall x\in A$に対し,$f(x,g(x))=0$となる$B$の点$g(x)$が唯一存在する。このとき$g:A\to\mathbb{R}^m$は$C^1$級。
	\end{thm}
\end{framed}

\begin{proof}
	$F:\mathbb{R}^n\times\mathbb{R}^m\to\mathbb{R}^n\times\mathbb{R}^m$を$F(x,y):=(x,f(x,y))$で定める。

    \[DF(x,y)=
    \left(
    \begin{array}{ccc|ccc}
        D_{x_1}F^1 & \cdots & D_{x_n}F^1 & D_{y_1}F^1 & \cdots & D_{y_m}F^1 \\
        D_{x_1}F^2 & \cdots & D_{x_n}F^2 & D_{y_1}F^2 & \cdots & D_{y_m}F^2 \\
        \vdots & \ddots & \vdots & \vdots & \ddots & \vdots \\
        D_{x_1}F^n & \cdots & D_{x_n}F^n & D_{y_1}F^n & \cdots & D_{y_m}F^n \\ \cline{1-6}
        D_{x_1}F^{n+1} & \cdots & D_{x_n}F^{n+1} & D_{y_1}F^{n+1} & \cdots & D_{y+m}F^{n+1} \\
        \vdots & \ddots & \vdots & \vdots & \ddots & \vdots \\
        D_{x_1}F^{n+m} & \cdots & D_{x_n}F^{n+m} & D_{y_1}F^{n+m} & \cdots & D_{y_m}F^{n+m} \\
    \end{array}
    \right)
    \]
    \[
    \begin{split}
        \det F'(a,b)&=\det
        \left(\begin{array}{c|c}
            E & 0 \\ \cline{1-2}
            * & M \\
        \end{array}\right) \\
        &=\det M \\
        &\neq 0
    \end{split}
    \]
	定理\ref{th2.11}より,$(a,b)$を含む開集合$V$,$(a,0)$を含む開集合$W$があって,$F:V\to W$は$C^1$級の逆関数$F^{-1}:W\to V$をもつ。$F(x,y)=(x,f(x,y))$だから$F^{-1}(x,y)=(x,k(x,y))$とおける。ここで$\pi:\mathbb{R}^n\times\mathbb{R}^m\to\mathbb{R}^m$を$\pi(x,y):=y$で定めると,$k=\pi\circ F^{-1}$となる。関数$k:W\to\mathbb{R}^m$は$C^1$級である。$\pi\circ F=f$も成立している。$\forall (x,y)\in W$に対し
	\begin{equation}\tag{*}
		\begin{split}
			f(x,k(x,y))&=f\circ F^{-1}(x,y)\\
			&=(\pi\circ F)\circ F^{-1}(x,y)\\
			&=\pi\circ(F\circ F^{-1})(x,y)\\
			&=\pi(x,y)\\
			&=y
		\end{split}
	\end{equation}
	$a$を含む開集合$A'\subset\mathbb{R}^n$,$b$を含む開集合$B\subset\mathbb{R}^m$を十分小さくとれば
	\[
		A'\times B\subset V
	\]
	\[
		A'\times\{0\}\subset W
	\]
	が成り立つ。そこで$x\in A'$ならば$(x,0)\in W$だから,$g(x):=k(x,0)$により関数$g:A'\to\mathbb{R}^m$が定義される。この$g$は作り方により$C^1$級で,$g(a)=b$である($\because F(a,b)=(a,f(a,b))=(a,0)$より,$(a,b)=F^{-1}(a,0)=(a,k(a,0))=(a,g(a))$)。$a$を含む開集合$A\subset A'\subset \mathbb{R}^n$を十分小さくとれば,$g(A)\subset B$となる。$x\in A$に対し,(*)より$f(x,k(x,0))=0$となり,$f(x,g(x))=0$が成立する。

	(一意性について)

	もう一つの関数$y_1=g_1(x)(f(x,g_1(x))=0)$があれば
	\[
	\begin{split}
		F(x,g_1(x))&=(x,f(x,g_1(x)))\\
		&=(x,0)\\
		&=F(x,g(x))
	\end{split}
	\]
	となるが,$F$は一対一写像なので$g_1(x)=g(x)$となる。

\end{proof}

\paragraph{陰関数の導関数について}

$f(x,g(x))=0$より$f^i(x,g(x))=0\ (1\leq i\leq m)$だから,両辺を$x_j$で偏微分すると(定理\ref{th2.9}などを用いて)
\[
0=D_jf^i(x,g(x))+\sum_{\alpha=1}^m D_{n+\alpha}f^i(x,g(x))\cdot D_jg^\alpha(x)\ (1\leq i\leq m,1\leq j\leq n)
\]
ここで$\det M\neq 0$だから,この連立方程式は$D_j g^\alpha(x)$に関して解ける。
