\section{可積分関数}
\paragraph{準備:有界関数の不連続度}
\begin{dfn*}[変動量]
$A\subset \mathbb{R}^n$とする。$f:A\to\mathbb{R}^n$:有界関数,$a\in A$とする。$\forall\delta>0$に対して
\[
	M(a,f,\delta):=\sup\{f(x):x\in A,|x-a|<\delta\}
\]
\[
	m(a,f,\delta):=\inf\{f(x):x\in A,|x-a|<\delta\}
\]
$\delta$を小さくすると$M(a,f,\delta)-m(a,f,\delta)(\geq 0)$も小さくなるので,$\displaystyle\lim_{\delta\to+0}\{M(a,f,\delta)-m(a,f,\delta)\}$が存在する。これを$f$の$a$での変動量といい,$\mathscr{O}(f,a)$とかく。
\end{dfn*}
\begin{thm*}
有界関数$f$が$a$で連続$\Leftrightarrow$$\mathscr{O}(f,a)=0$
\end{thm*}
\begin{proof}\
	\par\noindent\textbf{($\Rightarrow$)}\\
	$f$が$a$で連続とすると$\forall\epsilon>0,\exists\delta>0$\ s.t.\ $x\in A,|x-a|<\delta\Rightarrow|f(x)-f(a)|<\epsilon$。よって,$\forall x,y\in A:|x-a|<\delta,|y-a|<\delta$に対して
	\[
	|f(x)-f(y)|\leq |f(x)-f(a)|+|f(y)-f(a)|< 2\epsilon
	\]
	したがって
	\[
	M(a,f,\delta)-m(a,f,\delta)\leq 2\epsilon \footnotemark
	\]
	となり,$\mathscr{O}(f,a)=0$\\
	\par\noindent\textbf{($\Leftarrow$)}\\
	$\mathscr{O}(f,a)=0$とすると,$\forall\epsilon>0,\exists\delta>0$\ s.t.\ $M(a,f,\delta)-m(a,f,\delta)<\epsilon$となる。よって$\forall x\in A,|x-a|<\delta$に対して
	\[
		|f(x)-f(a)|\leq M(a,f,\delta)-m(a,f,\delta)<\epsilon
	\]
	となり連続。
\end{proof}
\footnotetext{
ここではsup-infのため等号が入る。
}
\begin{framed}
	\begin{thm}\label{th3.7}
		$A\subset\mathbb{R}^n$:閉集合,$f:A\to\mathbb{R}$:有界関数とする。このとき$\forall\epsilon>0$に対して,$B:=\{x\in A:\mathscr{O}(f,x)\geq \epsilon\}$は閉集合。
	\end{thm}
\end{framed}

\begin{proof}
	$\mathbb{R}^n-B(=B^c)$が開集合であることを示せばよい。
	\[
	x\in\mathbb{R}^n-B \Rightarrow
	\begin{cases}
 		(1) & x\notin A \\
 		& {\rm or}\\
 		(2) & x\in A かつ \mathscr{O}(f,x)<\epsilon
 	\end{cases}
	\]
	となる。\\
	(1)のとき,$A$は閉集合だから$x$を含む閉方体$C$が存在し,
	\[
		x\in C\subset \mathbb{R}^n-A\subset \mathbb{R}^n-B
	\]
	(2)のとき,十分小さい$\delta>0$を取ると$M(x,f,\delta)-m(x,f,\delta)<\epsilon$が成り立つ。そこで,$x$を含む開方体$C$を$y\in C$ならば$|x-y|<\delta$が成立するように小さくとる。このとき$\forall y\in C$に対し,十分小さい$\delta_1$をとって
	\[
		|z-y|<\delta_1 \Rightarrow |z-x|<\delta
	\]
	とできる。\footnotemark
	よって
	\[
		M(y,f,\delta_1)-m(y,f,\delta_1)<\epsilon
	\]
	したがって$\mathscr{O}(f,y)<\epsilon$が成り立ち,$C\in\mathbb{R}^n-B$となる。
\end{proof}
\footnotetext{
	中心$y$半径$\delta_1$の中に$z$があれば中心$x$半径$\delta$の中に$z$が含まれている。
}

\begin{framed}
	\begin{thm}\label{th3.8}
		$A\subset\mathbb{R}^n$:閉方体,$f:A\to\mathbb{R}$:有界関数,$f$の不連続点全体を$B$とする。このとき,$f:A$上可積分$\Leftrightarrow$$B$が測度$0$
	\end{thm}
\end{framed}

\begin{proof}\
	\par\noindent\textbf{($\Leftarrow$)}\\
	$B$が測度$0$とする。$\forall\epsilon>0$,$\exists$可算個の閉方体$c_i$\ ($i=1,2,\cdots$)\ s.t.\ (1)$\sum_{i=1}^\infty|c_i|<\epsilon$,(2)$c_i$の内部を$\overset{\circ}{c_i}$とするとき$B\subset\bigcup_{i=1}^\infty \overset{\circ}{c_i}$となる。\\
	一方$A-B$の各点$x$に対し,$x$を内部に含む閉方体$D_x$を十分小さくとると$M_{D_x}(f)-m_{D_x}(f)<\epsilon$が成り立つ。
    \footnote{
    $M_{D_x}(f)=\sup\{f(x):x\in D_x\}$\\
    $m_{D_x}(f)=\inf\{f(x):x\in D_x\}$\\
    Th3.7の前の内容
    }
	$\overset{\circ}{c_i}$および$\overset{\circ}{D_x}$の全体はコンパクト集合$A$の開被覆であるから,そのうちの有限個で$A$を覆うことができる。これに対して,$A$の分割$\Delta$を細かくとり,その小方体がすべて上記の有限個の開被覆の開方体のどれかに含まれるようにすることができる。\footnote{
    被った場合はどちらかに適当に割り振る
    }
	その中(分割された小方体)で$\overset{\circ}{c_i}$に含まれる小方体全体を$\mathscr{S}_1$,$\overset{\circ}{D_{x}}$に含まれる小方体全体を$\mathscr{S}_2$とする。このとき$|f(x)|$の$A$上の上限を$M$とすると
	\[
		\sum_{S\in\mathscr{S}_1}\{M_S(f)-m_S(f)\}|S|\leq 2M\epsilon
	\]
	\[
		\sum_{S\in\mathscr{S}_2}\{M_S(f)-m_S(f)\}|S|\leq \epsilon|A|
	\]
	が成立する。したがって,
	\[
	\begin{split}
	S(f,\Delta)-s(f,\Delta)&=\sum_{S}\{M_S(f)-m_S(f)\}|S| \\
	&\leq (2M+|A|)\epsilon
	\end{split}
	\]
	となり,$f$は可積分。
	\par\noindent\textbf{($\Rightarrow$)}\\
	$f$が$A$上可積分とする。$\displaystyle B_m:=\{x\in A:\mathscr{O}(f,x)\leq\frac{1}{m}\}$とおく。
    \footnote{
    測度論では常套手段。不連続なものというのはオシレーションが0より大きい。オシレーションが0なら連続。可算個でやりたい。任意の$m$でこれが示せればすべて測度0とわかりこれ自体が測度0となる。
    }
	(定理\ref{th3.7}より$B_m$は閉集合であり)$\displaystyle B=\bigcup_{m=1}^\infty B_m$だから,各$B_m$が測度0であることを示せばよい。(by定理\ref{th3.4})
	$\forall\epsilon>0$に対し,$A$の分割$\Delta$を$\displaystyle S(f,\Delta)-s(f,\Delta)<\frac{\epsilon}{m}$となるようにとる。$\Delta$の小方体のうち,その内部が$B_m$と共通点をもつようなものの全体を$\mathscr{S}$とする。$\mathscr{S}$は$\overset{\circ}{B_m}$
    \footnote{
    $B_m$から分割$\Delta$の各小方体の境界はすべて除いて考える。
    }
    を有限個で被覆している。$S\in\mathscr{S}$に対しては$\displaystyle M_S(f)-m_S(f)\geq\frac{1}{m}$
    \footnote{
    $\forall x\in\overset{\circ}{B_m}$の点のまわりが存在しているので変動量の$\frac{1}{m}$がでる。
    }
    となる。
	\[
	\begin{split}
		\frac{1}{m}\sum_{S\in\mathscr{S}}|S|
		&\leq\sum_{S\in\mathscr{S}}\{M_S(f)-m_S(f)\}|S| \\
		&\leq\sum_{S\in\Delta}\{M_S(f)-m_S(f)\}|S| \\
		&= S(f,\Delta)-s(f,\Delta) \\
		&\leq\frac{\epsilon}{m}
	\end{split}
	\]
	となり,$\displaystyle\sum_{S\in\mathscr{S}}|S|\leq\epsilon$となる。\\
	一方
    \footnote{
    境界に不連続点がたまたまあったとすると、差$M_S(f)-m_S(f)$が出ないため、内部のみを考えて、残りは後で考えている。
    }
    ,$\Delta$の小方体の境界全体は明らかに測度$0$である。つまり境界全体を覆う有限個の開方体$c_1,c_2,\cdots,c_l$で$\displaystyle \sum_{i=1}^l|c_i|<\epsilon$となるものが存在する
    \footnote{開方体$c_i$は細くとればいくらでも小さなものはとれる
    }
    。$\mathscr{S}$に属する$S$と,$c_i$の全体は$B_m$を覆う。このとき
	\[
	\sum_{S\in\mathscr{S}}|S|+\sum_{i=1}^l|c_i|<2\epsilon
	\]
	だから,$B_m$は測度$0$となる。
\end{proof}

\paragraph{一般の有界集合上の積分}
\begin{itemize}
	\item $A\subset\mathbb{R}^n$:部分集合に対し,関数$\chi_A:\mathbb{R}^n\to\mathbb{R}$を
	\[
		\chi_A(x):=
		\begin{cases}
			1 & x\in A\\
			0 & x\notin A
		\end{cases}
	\]
	によって定義する。これを$A$の特性関数\footnote{確率論での特性関数は、フーリエ変換をした後の関数のことを意味する、とかもあるので、数学の分野による名前のそれに注意。}という。
	\item $A$が有界ならば,$A$を含む閉方体$C$が存在する。関数$f:C\to\mathbb{R}$が有界で関数$f\cdot\chi_A$が$C$上可積分のとき,$f$は$A$上可積分といい,
	\[
		\int_A f := \int_C f\cdot\chi_A
	\]
	によって$f$の$A$上の積分を定める。\footnote{このcut offは画像処理などに応用されたりする。}特に$f$と$\chi_A$がともに$C$上可積分ならば$f\cdot\chi_A$も$C$上可積分であり,$f$は$A$上可積分となる。(可積分関数の積は可積分。)
\end{itemize}

\begin{note}
	$\displaystyle \int_A f$は$C$の取り方によらない。
\end{note}

\begin{framed}
	\begin{thm}
		$A\subset\mathbb{R}^n$:有界集合,$A$を含む閉方体を$C$とする。このとき,関数$\chi_A:C\to\mathbb{R}$が可積分$\Leftrightarrow$$A$の境界が測度$0$。ここで$A$の境界は有界閉集合,つまりコンパクト集合だから,測度$0$の代わりに容積$0$としても成立する。($\because$定理\ref{th3.6}による)
	\end{thm}
\end{framed}

\begin{proof}
	$A$の境界が関数$\chi_A$の不連続店の全体と一致することを示せば定理\ref{th3.8}よりOK。
	\begin{itemize}
		\item $x$が$A$の内部のとき$\exists U$:開方体 s.t. $x\in U\subset A$となる$U$が存在。$U$上で$\chi_A=1$より$\chi_A$は$x$で連続。
		\item $x$が$A$の外部のとき$\exists U$:開方体 s.t. $x\in U\subset \mathbb{R}^n-A$とできる。$U$上で$\chi_A=0$より$\chi_A$は$x$で連続。
		\item $x$が$A$の境界のとき$x$を含む任意の開方体$U$は$A$とも$\mathbb{R}^n-A$とも共通点をもつ。$y_1\in U\cap A$,$y_2\in U\cap(\mathbb{R}^n-A)$とすると,$\chi_A(y_1)=1$,$\chi_A(y_2)=0$より$\chi_A$は$x$では不連続。
	\end{itemize}
\end{proof}

\begin{dfn*}[Jordan可測集合]
	有界集合$A\subset\mathbb{R}^n$の境界が測度$0$のとき,$A$をJordan可測集合という。
\end{dfn*}
\begin{dfn*}[体積]
	関数$1$の$A$上積分$\displaystyle \int_A 1$を$A$の($n$次元)体積という。
\end{dfn*}
%\begin{note}
%3.8 そくどぜろならりーまんせきぶんできる。けど、ディリクレ関数はりーまんせきぶんできない。有理数で1、無理数で0 不連続点はすべての点。定理3.9によれば積分できないことがわかる。
%\end{note}
