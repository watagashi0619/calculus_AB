\section{偏導関数}
$f:\mathbb{R}^n\to\mathbb{R}$と$a\in\mathbb{R}^n$に対して
\[
\lim_{h\to0}\frac{f(a^1,\cdots,a^i+h,\cdots,a^n)-f(a^1,\cdots,a^i,\cdots,a^n)}{h}
\]
が存在するとき,$f$は$a$において,$x^i$について偏微分可能といい,極限値を偏微分係数といって$D_if(a)$とかく。
\footnote{
$g(x):=f(a^1,\cdots,a^{i-1},x^i,a^{i+1},\cdots,a^n)$\\
$g'(a^i)=\lim_{h\to0}\frac{g(a^i+h)-g(a^i)}{h}$
}
$f$が$A$上のすべての点で$x^i$について偏微分可能のとき,関数$D_if:A\to\mathbb{R}$を$f$の$x^i$に関する偏導関数という。
\paragraph{$x^i$に関する偏導関数の表し方}
\[
D_if(x),D_{x_i}f(x),f_{x_i}(x),\frac{\partial}{\partial x^i}f(x)
\]
など。

$D_if:\mathbb{R}^n\to\mathbb{R}$の$x^j$に関する偏導関数$D_j(D_if)(x)$のことを
\[
D_{i,j}f(x),D_{x^i,x^j}f(x),f_{x^i,x^j}(x),\frac{\partial^2}{\partial x^j \partial x^i}f(x)
など。
\]

次は$D_{j,i}=D_{i,j}$を保証する定理である。
\begin{framed}
	\begin{thm}\label{th2.5}
		点$a$を含むある$U$:open set上で$D_{i,j}f$と$D_{j,i}f$がともに連続ならば$D_{i,j}f(x)=D_{j,i}f(x)\ (x\ in\ U)$
	\end{thm}
\end{framed}
\begin{note}
証明は積分を用いて行う方が遥かに楽に示せるため後に回す。
\end{note}

\begin{framed}
	\begin{thm}\label{th2.6}
		$A\subset\mathbb{R}^n,f:A\to\mathbb{R}$が$A$の内部の点$a$で最大or最小となり,$D_if(a)$が存在するならば,$D_if(a)=0$
	\end{thm}
\end{framed}

\begin{proof}
	$g_i(x):=f(a_1,\cdots,x,\cdots,a^n)$とおくと,$g_i$は$a^i$で最大or最小となり,しかも$g_i$は$a^i$を含むある開区間で定義されている。$0=g'_i(a)=D_if(a)$となる。	(最後の等号は偏微分の定義による。)
\end{proof}

\begin{framed}
	\begin{thm}
		$f:\mathbb{R}^n\to\mathbb{R}^m$が点$a$で全微分可能ならば,偏微分$D_jf^i(a)$ $(1\leq i \leq m,1\leq j \leq n)$が存在し,ヤコビ行列$Df(a)$は$m\times n$行列$(D_jf^i(a))_{ij}$に等しい。
	\end{thm}
\end{framed}

\begin{proof}
	\subparagraph{$m=1$の場合}
	$f:\mathbb{R}^n\to\mathbb{R}$となる。関数$h:\mathbb{R}\to\mathbb{R}^n$を$h(x):=(a^1,\cdots,x,\cdots,a^n)$と定める(ただし$x$は$j$番目成分である)。このとき,$D_if(a)=(f\circ h)'(a^j)$であり,定理\ref{th2.2}により
	\[
	\begin{split}
	(f\circ h)'(a^j)&=f'(a)\cdot h'(a^j)\\
	&=f'(a)\cdot\begin{pmatrix} 0 \\ \vdots \\ 1 \\ \vdots \\ 0\end{pmatrix}
	\end{split}
	\]
	となる(ただし$1$となっているのは第$j$番目成分)。
	これは$D_if(a)$が存在し,それが$1\times n$行列$f'(a)$の第$j$成分であることを示している。\footnotemark
	\subparagraph{一般の$m$に対する場合}
	定理\ref{th2.3}より,$f^i$は全微分可能で$(f^i)'(a)$\footnotemark は$f'(a)$の第$i$行になっているので,OK
\end{proof}
\footnotetext{
$m=1$のとき$f'(a)=(D_1f(a),D_2f(a),\cdots,D_nf(a))$
}
\footnotetext{
$f^i=\pi^i\circ f,\pi^i:\mathbb{R}^m\to\mathbb{R}(\pi^i(x)=x^i)$\\
これより$(f^i)'(a)=\pi^i\circ f'(a)$
}

\begin{framed}
	\begin{thm}\label{th2.8}
		$f:\mathbb{R}^n\to\mathbb{R}^m$に対し,偏導関数$D_jf^i(x)$がすべて存在し,それらがすべて点$a\in\mathbb{R}^n$で連続ならば,全微分$Df(a)$が存在する。
	\end{thm}
\end{framed}

\begin{proof}
	$m=1$の場合を考えれば十分\footnotemark。$f:\mathbb{R}^n\to\mathbb{R}$とする。
	\[
	\begin{split}
	f(a+h)-f(a)&=f(a^1+h^1,a^2,\cdots,a^n)-f(a^1,\cdots,a^n)\\
	&\ \ \ \ +f(a^1+h^1,a^2+h^2,a^3,\cdots,a^n)-f(a^1+h^1,a^2,\cdots,a^n)\\
	&\ \ \ \ +\ \ \cdots \\
	&\ \ \ \ +f(a^1+h^1,\cdots,a^{n-1}+h^{n-1},a^n+h^n)-f(a^1+h^1,\cdots,a^{n-1}+h^{n-1},a^n)
	\end{split}
	\]
	$D_1f$が関数$g(x)=f(x,a^2,\cdots,a^n)$の導関数であることから,$g$に平均値の定理を使うと,$f(a^1+h^1,a^2,\cdots,a^n)-f(a^1,\cdots,a^n)=h^1\cdot D_1f(c_1,a^2,\cdots,a^n)$となるような$c_1$が$a^1$と$a^1+h^1$の間に存在する。同様に,はじめの式の右辺第$i$項目について,$f(a^1+h^1,\cdots,a^i+h^i,a^{i+1},\cdots,a^n)-f(a^1+h^1,\cdots,a^{i-1}+h^{i-1},a^i,\cdots,a^n)=h^i\cdot D_if(a^1+h^1,\cdots,a^{i-1}+h^{i-1},c_i,a^{i+1},\cdots,a^n)$となる$c_i$が$a^i$と$a^i+h^i$の間に存在する。(ここで$D_if(c_i):= D_if(a^1+h^1,\cdots,a^{i-1}+h^{i-1},c_i,a^{i+1},\cdots,a^n)$とおく。)ゆえに
	\[
	\begin{split}
		\lim_{h\to0}\frac{\left|f(a+h)-f(a)-\sum_{i=1}^nD_if(a)h^i\right|}{|h|}
		&=\lim_{h\to0}\frac{\left|\sum_{i=1}^n\left\{D_if(c_i)-D_if(a)\right\}h^i\right|}{|h|}\\
		&\leq\lim_{h\to0}\sum_{i=1}^n\left\{D_if(c_i)-D_if(a)\right\}\frac{|h^i|}{|h|}\\
		&\leq\lim_{h\to0}\sum_{i=1}^n|D_if(c_i)-D_if(a)|\\
		&=0\footnotemark
	\end{split}
	\]
\end{proof}

	\footnotetext{この証明が可能であると仮定すれば、2.3から各成分関数が全微分可能とできるため。}
	\footnotetext{
	$f'(a)=(D_1f(a),\cdots,D_nf(a)$\\
	三角不等式を用いている。また,最後の等号は$D_if$は$a$で連続より,$h\to0$のとき$c_i\to h^i$となることによる。
	}

\paragraph{$C^k$級関数}
$f:\mathbb{R}^n\to\mathbb{R}^m$について,すべての偏導関数$D_jf^i(x)$が存在し,$a\in\mathbb{R}^n$でそれらがすべて連続なとき$f$は$a$で$C^1$級という。同様に,$k$階までの 偏導関数$D_{i_1,\cdots,i_k}f(x)$が存在し,それらがすべて$a\in\mathbb{R}^n$で連続なとき,$f$は$a$で$C^k$級という。定理\ref{th2.5}より,$C^k$級関数は微分の順序には偏導関数は依存しない。

\begin{framed}
	\begin{thm}\label{th2.9}
		$g_1,\cdots,g_m:\mathbb{R}^n\to\mathbb{R}$が点$a\in\mathbb{R}^n$で$C^1$級,$f:\mathbb{R}^m\to\mathbb{R}$が点$(g_1(a),\cdots,g_m(a))$で全微分可能とする。
		関数$F:\mathbb{R}^n\to\mathbb{R}$を$F(x):=f(g_1(x),\cdots,g_m(x))$で定めると
		\[
		D_iF(a)=\sum_{j=1}^mD_jf(g_1(a),\cdots,g_m(a))\cdot D_ig_j(a)
		\]
	\end{thm}
\end{framed}

\begin{proof}
	$g:=(g_1,\cdots,g_m)$とすれば,$F=f\circ g$と書ける。各$g_i$は$a$で$C^1$級より,定理\ref{th2.8}から$g$は$a$で全微分可能。定理$\ref{th2.2}$により

	\[
	\begin{split}
	F'(a)&=f'(g(a))\cdot g'(a)\\
	&=(D_1f(g(a)),\cdots,D_mf(g(a)))\cdot
	\begin{pmatrix}
		D_1g_1(a) & \cdots & D_ng_1(a)\\
		\vdots & \ddots & \vdots \\
		D_1g_m(a) & \cdots & D_ng_m(a)
	\end{pmatrix}
	\end{split}
	\]

この左辺の第$i$成分は$D_iF(a)$であり,右辺の第$i$成分は$\displaystyle\sum_{j=1}^m D_jf(g_1(a),\cdots,g_m(a))\cdot D_ig_j(a)$となりOK。
\end{proof}

\begin{framed}
	\begin{lem}[リプシッツ連続\footnotemark]\label{lem2.10}
		$A\subset\mathbb{R}^n:$閉方体,$f:A\to\mathbb{R}^n:C^1$級,$\exists M>0$\\ s.t. $|D_jf^i(x)|\leq M(\forall x\in A) \Rightarrow |f(x)-f(y)|\leq n^2 M|x-y|\ (\forall x,y\in A)$
	\end{lem}
\end{framed}
\footnotetext{
リプシッツ連続という概念は,例えばディープラーニングで使われる活性化関数ReLUなどで用いられる。
}
\begin{proof}
	$f=(f^1,\cdots,f^n)$に対して
	\[
	f^i(y)-f^i(x)=\sum_{j=1}^n\left\{f^i(y^1,\cdots,y^j,x^{j+1},\cdots,x^n)-f^i(y^1,\cdots,y^{j-1},x^j,\cdots,x^n)\right\}
	\]
	ここで,平均値の定理より
	\[
	\exists z_{ij}\in A\ {\rm s.t.}\ f^i(y^1,\cdots,y^j,x^{j+1},\cdots,x^n)-f^i(y^1,\cdots,y^{j-1},x^j,\cdots,x^n)=(y^j-x^j)D_jf^i(z_{ij})
	\]
	よって
	\[
	|f^i(y)-f^i(x)|\leq\sum_{j=1}^n|y^i-x^i|M\leq nM|y-x|\footnotemark
	\]
	したがって
	\[
	|f(y)-f(x)|\leq\footnotemark\sum_{i=1}^n|f^i(y)-f^i(x)|\leq n^2M|y-x|
	\]
\end{proof}

\footnotetext{
$\because |y^i-x^i|\leq|y-x|$
}
\footnotetext{
$\sqrt{a_1^2+\cdots+a_n^2}\leq|a_1|+\cdots+|a_n|$
}

\begin{framed}
	\begin{thm}[逆関数の定理]\label{th2.11}
$f:\mathbb{R}^n\to\mathbb{R}^n$は点$a\in\mathbb{R}^n$を含む開集合で$C^1$級かつ$\det f'(a)\neq 0$\footnotemark であるとする。このとき,

		$\exists V:a$を含む開集合,$\exists W:f(a)$を含む開集合s.t.$f:V\to W$が連続な逆関数$f^{-1}:W\to V$をもつ。\\
		この$f^{-1}$は$C^1$級であり,$(f^{-1})'(y)=(f'(f^{-1}(y)))^{-1}(\forall y\in W)$が成立する。\\
		特に$f$が$C^\infty$級ならば$f^{-1}$も$C^\infty$級である。
	\end{thm}
\end{framed}

\footnotetext{$\det f'(a)$をヤコビアンという。}

\begin{proof}(Step1.〜Step.7)
	\begin{enumerate}\renewcommand{\labelenumi}{Step\arabic{enumi}.}
	\item 線形写像$Df(a)$を$\lambda$とかくと,$\det f'(a)\neq0$より$\lambda$は正則である。
	\[
	\begin{split}
	D(\lambda^{-1}\circ f)(a)
	&=D(\lambda^{-1})(f(a))\circ Df(a) \\
	&=\footnotemark\lambda^{-1}\circ Df(a) \\
	&=\lambda^{-1}\circ\lambda \\
	&=id\footnotemark
	\end{split}
	\]
	\footnotetext{
	この定理が成立すると仮定したときに,$(f^{-1})(y)=(f'(f^{-1}(y)))^{-1}$が成り立つので,これの$y=f(a)$とすれば,$D(\lambda^{-1})(f(a))=Df^{-1}(f(a))=(f^{-1})'(f(a))=(f'(f^{-1}(f(a))))^{-1}=(f'(a))^{-1}=\lambda^{-1}$
	}
	\footnotetext{$id$は恒等写像(identity mapping)の意。}
	これは恒等写像である。
	$\lambda^{-1}\circ f$に対して定理が成立すれば$f$に対しても成立する。よって,$\lambda$が恒等写像であると仮定しても一般性を失わない。
	\item もし$f(a+h)=f(a)$とすると
	\[
	\begin{split}
	\frac{|f(a+h)-f(a)-\lambda(h)|}{|h|}&=\frac{|\lambda(h)|}{|h|}\\
	&=\frac{|h|}{|h|}\\
	&=1
	\end{split}
	\]
	となるが,$f(a)$は$a$で微分可能より
	\[
	\lim_{h\to0}\frac{|f(a+h)-f(a)-\lambda(h)|}{|h|}=0
	\]
	よって$a$に十分近く,$a$自身ではない点$x$に対しては$f(x)=f(a)$はならない。したがって,$a$をその内部に含む\footnote{内部であって境界でない。}閉方体$U$で次の条件(1)をみたすものがとれる。 
	\setcounter{equation}{0}
	\begin{equation}
		x\in U,x\neq a\Rightarrow f(x)\neq f(a)
	\end{equation}
	さらに,$f$は$a$を含むある開集合で$C^1$級だから$U$として次の条件(2)(3)も満たすものが取れる。
	\begin{equation}
		x\in U\Rightarrow \det f'(x)\neq0
	\end{equation}
	\begin{equation}
	x\in U\Rightarrow|D_jf^i(x)-D_jf^i(a)|\leq\frac{1}{2n^2}
	\end{equation}
	(3)により補題\ref{lem2.10}を関数$g(x):=f(x)-x$に適用でき,その結果
	\[
	|f(x_1)-x_1-(f(x_2)-x_2)|\leq\frac{1}{2}|x_1-x_2|\ (\forall x_1,x_2\in U)
	\] \footnotemark
	\[
	\begin{split}
		|x_1-x_2|-|f(x_1)-f(x_2)|&\leq|f(x_1)-x_1-(f(x_2)-x_2)|\\
		&\leq\frac{1}{2}|x_1-x_2|
	\end{split}
	\]\footnotemark
	よって
	\begin{equation}
		|x_1-x_2|\leq2|f(x_1)-f(x_2)|\ (\forall x_1,x_2\in U)
	\end{equation}
	\footnotetext{Lipschitz連続の逆
	$f$は$Df(a)=\lambda=id$より$f'(a)=E$($E$は単位行列)。$|D_if'(a)|_{x\in U}\leq 2$。$|g(x_1)-g(x_2)|\leq n^2M|x_1-x_2|$の$M=2$くらい。$|D_jg^i(x)|\leq M=\frac{1}{2n^2}$とする。$|D_jg^i(x)|=|D_jf^i(x)-D_jx^i|=|D_jf^i(x)-\delta_{ij}|=|D_jf^i(x)-D_jf^i(a)|$($Df(a)=id$ゆえ$D_jf^i(a)=\delta_{ij}$)。
	。逆関数がリプシッツ連続と言っている。
	}
	\footnotetext{三角不等式による。}
	\item\footnote{逆関数をつくるステップ}
	$U$の境界を$B$とかくと,定理\ref{th1.9}によって$f(B)$はコンパクト集合であり,性質(1)によって$f(B)$は$f(a)$を含まない。したがって$\exists d>0$\ s.t.\ $|f(a)-f(x)|\geq d\ (\forall x\in B)$そこで$\displaystyle W:=\left\{y:|y-f(a)|\leq\frac{d}{2}\right\}$とおくと
	\begin{equation}
		y\in W,x\in B\Rightarrow |y-f(a)|<|y-f(x)|
	\end{equation}
が成立する。

	\item $\forall y\in W$に対して,$U$の内部$U-B$の点$x$で,$f(x)=y$となるものが唯一存在することを示す。
	$g:U\to\mathbb{R}$を
	\[
	g(x):=|y-f(x)|^2=\sum_{i=1}^{n}(y^i-f^i(x))^2
	\]
	と定める。$g$は連続だから,コンパクト集合$U$の上で最小値を取る。ところが(5)により$x\in B$なら$g(a)<g(x)$だから最小値を取る点は$B$に属さない。最小値をとる点を$x_0$とすると,$x_0\in U-B$である。定理\ref{th2.6}より$\forall j=1,\cdots,n$に対して$D_jg(x_0)$すなわち
	\[
	\sum_{i=1}^n 2(y^i-f^i(x_0))\cdot D_jf^i(x_0)=0\ (1\leq j\leq n)
	\]
	性質(2)により$(D_jf^i(x_0))_{ij}$は正則であるから($\det(D_jf^i(x_0))\neq0$)逆行列が存在する。\footnote{上の式に$(D_jf^i(x_0))_{ij} $の逆行列を作用させて残るのは$y^i-f^i(x_0)$の部分だけで,さらに左辺は0である。}よって$\forall i=1,\cdots,n$に対して$y^i-f^i(x_0)=0$となる、ゆえに$y=f(x_0)$となる。また,このような$x_0$が唯一つであることは(4)より明らか。

	\item $V:=(U-B)\cap f^{-1}(W)$\footnote{$f$は$a$の近傍についてしか条件を考えていないため,それ以外のところから$W$に移す写像になっている可能性があるため,$a$近傍についてのみ議論するために,$f^{-1}(W)$に属しているだけでなく$U-B$に属するという条件も付加している。}
	とおくと$V$は$a$を含む開集合で関数$f:V\to W$は逆関数$f^{-1}:W\to V$をもつことがわかる。性質(4)を書き換えると,$y_1:=f(x_1),f_2:=f(x_2)$とすると$y_1=f^{-1}(y_1),x_2=f^{-1}(y_2)$で
	\begin{equation}
	y_1,y_2\in W\Rightarrow |f^{-1}(y_1)-f^{-1}(y_2)|\leq 2|y_1-y_2|
	\end{equation}
	となり,これは$f^{-1}$が連続であることを示す。

	\item\footnote{$C^1$を示す}
	$f^{-1}$の微分可能性と導関数の表示を求める。
	$x\in V$に対し,$\mu:=Df(x)$とおく。$f^{-1}$が$y=f(x)$で全微分可能でその値が$\mu^{-1}$であることを示す。定理\ref{th2.2}の証明と同様に,$x_1\in V$に対して
	\begin{equation}\tag{*}
		f(x_1)=f(x)+\mu(x_1-x)+\phi(x,x_1)
	\end{equation}
	\begin{equation}\tag{**}
	\lim_{x_1\to x}\frac{|\phi(x,x_1)|}{|x_1-x|}=0\footnotemark
	\end{equation}
	が成り立つ。
	\footnotetext{
	$\displaystyle\lim_{x_1\to x}\frac{|f(x_1)-f(x)-\mu(x_1-x)|}{|x_1-x|}=0$
	}
	(*)の両辺に$\mu^{-1}$を作用させると
	\[
	\mu^{-1}(f(x_1)-f(x))=x_1-x+\mu^{-1}(\phi(x,x_1))
	\]
	となる。$y_1:=f(x_1)$,$y=f(x)$とおくと$x_1=f^{-1}(y_1)$,$x=f^{-1}(y)$より
	\begin{equation}\tag{*'}
		f^{-1}(y_1)=f^{-1}(y)+\mu^{-1}(y_1-y)-\mu^{-1}(\phi(f^{-1}(y),f^{-1}(y_1)))
	\end{equation}
	よって示すべきことは
	\[
	\lim_{y_1\to y}\frac{|\mu^{-1}(\phi(f^{-1}(y),f^{-1}(y_1)))|}{|y_1-y|}=0
	\]
	である。ここで,線形写像の有界性($|\mu^{-1}(z)|\leq\exists M|z|$)より
	\[
	\lim_{y_1\to y}\frac{|\phi(f^{-1}(y),f^{-1}(y_1))|}{|y_1-y|}=0
	\]
	を示せばよい。このとき
	\[
		\frac{|\phi(f^{-1}(y),f^{-1}(y_1))|}{|y_1-y|}=\frac{|\phi(f^{-1}(y),f^{-1}(y_1))|}{|f^{-1}(y_1)-f^{-1}(y)|}\cdot\frac{|f^{-1}(y_1)-f^{-1}(y)|}{|y_1-y|}
	\]
	$f^{-1}$は連続だから$y_1\to y \Rightarrow f^{-1}(y_1)\to f^{-1}(y)$となる。よって,$\displaystyle \frac{|\phi(f^{-1}(y),f^{-1}(y_1))|}{|f^{-1}(y_1)-f^{-1}(y)|}$は(**)より$0$に近づく。$\displaystyle \frac{|f^{-1}(y_1)-f^{-1}(y)|}{|y_1-y|}$は(6)により定数$2$で抑えられるので,結局
	\[
	\lim_{y_1\to y}\frac{|\phi(f^{-1}(y),f^{-1}(y_1))|}{|y_1-y|}=0
	\]
	となる。

	\item $f^{-1}$が$C^1$級であること,および$f$が$C^\infty$級ならば$f^{-1}$も$C^\infty$級を示す。\\
	$f^{-1}$の全微分の行列$(f^{-1})'(x)$は$f$の全微分の行列$f'(x)$の逆行列であり,それはCramerの公式により,分母は$\det(D_jf^i(x))$,分子は$D_jf^i(x)\ (1\leq i,j\leq n)$の多項式である。よって$(f^{-1})'$は連続である\footnote{$f$は$C^1$級であることによる。}。また,$f$が$C^\infty$級なら$D_jf^i(x)$はすべて$C^\infty$級だから$(f^{-1})'$も$C^\infty$級である。
	\end{enumerate}
\end{proof}
