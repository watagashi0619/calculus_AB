\section{変数変換}
\paragraph{1次元}
$f:\mathbb{R}\to\mathbb{R}:$連続関数に対して,変換$g:[a,b]\to\mathbb{R}:C^1$級関数\\
このとき
\[
	\int_{g(a)}^{g(b)}f = \int_a^b(f\circ g)\cdot g'
\]
(変数変換)が成立する。$g$が1対1ならば
\[
	\int_{g([a,b])}f=\int_{[a,b]}(f\circ g)\cdot|g'|
\]
とかける。

\paragraph{多次元での変数変換}\

\begin{framed}
	\begin{thm} \label{th3.11}
		$A$を$\mathbb{R}^n$の開集合,$g:A\to\mathbb{R}^n$を1対1で$C^1$級関数,すべての$x\in A$で$\det g'(x)\neq 0$となるものとする。$f:g(A)\to\mathbb{R}$が可積分であれば
		\[
			\int_{g(A)}f=\int_A(f\circ g)\cdot|\det g'|
		\]
		が成立する。
	\end{thm}
\end{framed}

\begin{proof}
	は別途†闇のpdf†にて。また,これに関するレポート課題を(上位者救済措置として?)出す。
\end{proof}

\begin{example}[極座標変換:$x_1=r\cos\theta,x_2=r\sin\theta$]
これは$\mathbb{R}^2\leftrightarrow [0,\infty)\times[0,2\pi]$の対応で,境界を除き1対1,$g(r,\theta)={}^t(r\cos\theta,r\sin\theta)=(x_1,x_2)$とすると
\[
	g'=
	\begin{pmatrix}
	\cos\theta & -r\sin\theta \\
	\sin\theta & r\cos\theta
	\end{pmatrix}
\]
より,$|\det g'|=r$は$r=0$を除いて$0$でない。この変換で$g(A)$と$A$が対応しているとき
\[
	\int_{g(A)}f(x,y)dxdy=\int_A f(r\cos\theta,r\sin\theta)rdrd\theta
\]
となる。
\begin{ex}
	3次元球の体積$\displaystyle V=2\int_{x^2+y^2\leq a^2}\sqrt{a^2-x^2-y^2}dxdy$
\end{ex}
円$x^2+y^2\leq a^2$は極座標変換により,$0\leq r\leq a,0\leq \theta\leq 2\pi$に対応するので,
\[
\begin{split}
	V&=2\int_{\theta=0}^{\theta=\pi}\left(\int_{r=0}^{r=a}\sqrt{a^2-r^2}rdr\right)d\theta\\
	&=4\pi\int_0^a r\sqrt{a^2-r^2}dr\\
	&=\frac{4}{3}\pi a^3
\end{split}
\]

\begin{framed}
	\begin{thm}[Sardの定理]
		$A$が$\mathbb{R}^n$の開集合,$g:A\to\mathbb{R}^n$が$C^1$級のとき$B:=\{x\in A:\det g'(x)=0\}$とおくと,$g(B)$は($n$次元)測度$0$である。\footnotemark
	\end{thm}
\end{framed}
\footnotetext{
この定理によって定理\ref{th3.11}における$\det g'(x)\neq 0$という仮定はなくてもよいことがわかる。
}
\end{example}