
\section{$\mathbb{R}^n$の開集合・閉集合・コンパクト集合}
\begin{note}
本格的に勉強がしたければ,集合と位相の本をやるとよい。
\footnote{松坂位相とか内田位相とか。松坂位相の第4章,第5章にここに書いてあるような内容が載ってたりする。}
\end{note}
$A_m\subset\mathbb{R}^n\ (m=1,2,\cdots)$とする。
\begin{itemize}
\item 和集合(合併)
	\[
	\bigcup_{m=1}^\infty A_m:=\{x\in\mathbb{R}^n:\exists m\in\mathbb{N}\ {\rm s.t.}\ x\in A_m\}
	\]
\item 共通部分
	\[
	\bigcap_{m=1}^\infty A_m:=\{x\in\mathbb{R}^n:\forall m\in\mathbb{N}, x\in A_m\}
	\]
\end{itemize}
集合$A\subset\mathbb{R}^m$と$B\subset\mathbb{R}^n$に対し,
\[
A\times B:=\{(x,y)\in\mathbb{R}^{m+n}:x\in A,y\in B\}
\]
\begin{example}\
\begin{itemize}
	\item $\mathbb{R}^{m+n}=\mathbb{R}^m\times\mathbb{R}^n$
	\item $[a,b]\times [c,d]=\{(x,y)\in\mathbb{R}^2:x\in[a,b],y\in[c,d]\}$\\
\end{itemize}
\end{example}

\begin{note}
一般に
\begin{itemize}
    \item $[a_1,b_1]\times[a_2,b_2]\times\cdots\times[a_n,b_n]\subset\mathbb{R}^n$の形の集合を$\mathbb{R}^n$の閉方体という
    \item $(a_1,b_1)\times(a_2,b_2)\times\cdots\times(a_n,b_n)\subset\mathbb{R}^n$の形の集合を$\mathbb{R}^n$の開方体という
\end{itemize}
\end{note}

\begin{framed}
\begin{dfn*}[開集合\footnotemark]
集合$U\subset\mathbb{R}^n$が開集合$\overset{def}{\Leftrightarrow}$$\forall x\in U$に対して$x$を含み,かつ,$U$に含まれる開方体\footnotemark が存在する。
\end{dfn*}
\end{framed}

\footnotetext{本やネットで調べてみると開集合を開方体で定義しているものは(編集者が探した感じでは)見つからなかったが,このように定義しておくことで後でリーマン積分を定義するときにやりやすくなるんだそうです(とtwitterでプロに教えてもらいました)。}
\footnotetext{この開方体は$x$に依存する。}

\begin{example} 
\begin{itemize}
	\item 開方体は開集合
	\item $\{|x|<1\}$:ballは開集合(一般に集合$\{x\in\mathbb{R}^n:|x-a|<r\}$は開集合)
	\item $\mathbb{R}^n$全体は開集合
\end{itemize}

\end{example}
\begin{framed}
\begin{dfn*}[閉集合]
$C\subset\mathbb{R}^n$が閉集合$\overset{def}{\Leftrightarrow}$$\mathbb{R}^n-C:=\{x\in\mathbb{R}^n:x\notin C\}$が開集合
\end{dfn*}
\end{framed}
集合$A\subset\mathbb{R}^n$と点$x\in\mathbb{R}^n$の関係は次の3つのいずれかとなる。
\begin{enumerate}
	\item $x\in B\subset A$となる開方体$B$が存在する。
	\item $x\in B\subset \mathbb{R}^n-A$となる開方体$B$が存在する。
	\item $x\in B$となる開方体は$A$の点と$\mathbb{R}^n-A$の点を少なくとも1つずつ含む。
\end{enumerate}
集合$A$に対し,
\begin{enumerate}
\item を満たす点全体を$A$の内部という。
\item を満たす点全体を$A$の外部という。
\item を満たす点全体を$A$の境界という。
\end{enumerate}
\begin{note}
$A$の内部は開集合,$A$の外部は開集合となる。よってその残りである$A$の境界は閉集合となる。開集合の和集合は開集合である。
\end{note}
$\mathscr{O}$を開集合の族とする。(i.e. $\mathscr{O}=\{U_{\lambda}\subset\mathbb{R}^n:U_\lambda$はopen,$\lambda\in\Lambda\}$)
\begin{dfn*}[開被覆 over covering]
    $\mathscr{O}$が$A\subset\mathbb{R}^n$の開被覆(open covering)であるとは,任意の$x\in A$に対して$\mathscr{O}$の中の開集合$U_\lambda$があって$x\in U_\lambda$であることである。
\end{dfn*}
\begin{dfn*}[コンパクト compact]
集合$A\subset\mathbb{R}^n$がコンパクト(compact)であるとは,$A$の任意の開被覆$\mathscr{O}$に対して$\mathscr{O}$の中の有限個の開集合をうまく選べば,それだけで$A$を覆うことができることである。
\end{dfn*}
\begin{example}\
	\begin{itemize}
		\item 有限個の点の集合はコンパクト
		\item $\displaystyle\left\{0と\frac{1}{n}の全部(nは自然数)\right\}$は$\mathbb{R}$のコンパクト集合
		\item $\displaystyle\left\{\frac{1}{n}の全部(nは自然数)\right\}$は$\mathbb{R}$のコンパクト集合でない
	\end{itemize}
\end{example}

\begin{itemize}
\item 集合$A\subset\mathbb{R}^n$が有界である$\Leftrightarrow$$\exists M>0\ \rm{s.t.}\ A\subset\{x\in\mathbb{R}^n:|x|<M\}$
\end{itemize}

\newpage

\begin{framed}
\begin{thm}[Heine-Borel]
    閉区間はcompactである。
\end{thm}
\end{framed}

\begin{proof}
$\mathscr{O}$を閉区間$[a,b]$の開被覆とする。$x\in[a,b]$で$[a,x]$が$\mathscr{O}$の中の有限個だけで覆われるものの全体を$A$とする($A:=\{x\in[a,b]:[a,x]が\mathscr{O}の中の有限個で覆われる\}$)。明らかに$a\in A$であり,$A$は上に有界である(例えば$b$が一つの上界)。compactの定義より$b\in A$を示せばよい。そこで,$A$の上限を$\alpha$とし,
\begin{enumerate}
\item $\alpha\in A$
\item $b=\alpha$
\end{enumerate}
を示せばよい。
\begin{enumerate}
\item $\mathscr{O}$は$[a,b]$の開被覆であり,$a\leq b$だから$a\in U$となる開集合$U\in\mathscr{O}$が存在する。\\
$\alpha$は$A$の上限なので,$\alpha$の十分近くに$\exists x\in A$\ \rm{s.t.}\ $x\in U$となるものがある。$x\in A$より$[a,x]$は$\mathscr{O}$の中の有限個で覆われている。また$[x,\alpha]$は1個の開集合$U\in\mathscr{O}$で覆われている。よって$[a,\alpha]=[a,x]\cup[x,\alpha]$は$\mathscr{O}$の有限個で覆われる。したがって$\alpha\in A$

\item $\alpha<b$と仮定する。このとき$\alpha<x'<b$となる$x'$で$U$に属するものが存在する。$\alpha\in A$だから$[a,\alpha]$は$\mathscr{O}$の有限個で覆われている。$[\alpha,x']$も1個の開集合$U\in\mathscr{O}$で覆われている。よって$x'\in A$となり,$\alpha$が$A$の上限であることに矛盾。したがって$\alpha=b$。
\end{enumerate}

\end{proof}

\begin{itemize}
\item $B\subset\mathbb{R}^m$がcompactで$x\in\mathbb{R}^n$ならば$\{x\}\times B\subset\mathbb{R}^{n+m}$もcompact。
\end{itemize}

\begin{framed}
\begin{thm}\label{th1.4}
$B\subset\mathbb{R}^m$はcompact,点$x\in\mathbb{R}^n$に対して$\mathscr{O}$を$\{x\}\times B\subset\mathbb{R}^{n+m}$の開被覆とする。このとき,ある開集合$U\subset\mathbb{R}^n$であって$x\in U$かつ$U\times B$は$\mathscr{O}$の中の有限個で覆われるようなものが存在する。
\end{thm}
\end{framed}

\begin{proof}
    $\{x\}\times B$がcompactより,有限個の開被覆($\mathscr{O}'$とする)を$\mathscr{O}$から選んで$\{x\}\times B$がそれ($\mathscr{O}'$)で覆える。よって$U\times B$が$\mathscr{O}'$で覆われるような開集合$U$を見つければよい。
    $\forall y\in B$に対して$\exists W\in\mathscr{O}'$\ s.t.\ $(x,y)\in W$($\because (x,y)\in\{x\}\times B$)。$W$はopenより$\exists U_y\times V_y:$開方体 s.t. $(x,y)\in U_y\times V_y\subset W$。ここで$\{V_y\}_{y\in B}$は$B$の開被覆で$B$はcompactより有限個の$V_y$で$B$を覆うことができる。
    \[
        B\subset V_{y_1}\cup V_{y_2}\cup\cdots\cup V_{y_k}
    \]
    そこで$U:=U_{y_1}\cap U_{y_2}\cap\cdots\cap U_{y_k}$とおくと,$U$は開方体で,$\forall (x',y')\in U\times B$に対して$y'$はある$i$に対して$y'\in V_{y_i}$であり,かつ$x'\in U_{y_i}$となる。よって$(x',y')\in U_{y_i}\times V_{y_i}$となり,$U_{y_i}\times V_{y_i}$はある$W\in\mathscr{O}'$に含まれる。
\end{proof}

\begin{framed}
\begin{cor}
$A\subset\mathbb{R}^n,B\subset\mathbb{R}^m$が共にcompactならば$A\times B\subset\mathbb{R}^{m+n}$もcompact。
\end{cor}
\end{framed}

\begin{proof}
    $\mathscr{O}$を$A\times B$の開被覆とすると,$\forall x\in A$に対し$\mathscr{O}$は$\{x\}\times B$を覆う。定理\ref{th1.4}より$\exists U_x\subset\mathbb{R}^n:$open s.t. $x\in U_x$かつ$U_x\times B$は$\mathscr{O}$の有限個で覆われる。$A$はcompactで$\{U_x\}_{x\in A}$は$A$の開被覆だから,その中の有限個$U_{x_1},U_{x_2},\cdots,U_{x_k}$がすでに$A$を覆う。各$U_{x_i}\times B$は$\mathscr{O}$の中の有限個で覆われるので,$A\times B$全体が$\mathscr{O}$の中の有限個で覆われる。($A\subset U_{x_1}\cup U_{x_2}\cup \cdots \cup U_{x_k}$)
\end{proof}

\begin{framed}
    \begin{cor}\label{cor1.6}
        各$A_i$がcompactならば$A_1\times A_2\times \cdots \times A_k$もcompactである。特に$\mathbb{R}^k$の閉方体はcompactである。
    \end{cor}
\end{framed}

\begin{framed}
    \begin{cor}
        $\mathbb{R}^n$の有界閉集合はcompact(逆も成立)
    \end{cor}
\end{framed}

\begin{proof}
    $A\subset\mathbb{R}^n$が有界閉集合ならば$A$を含む閉方体$B$が存在する。$\mathscr{O}$をAの開被覆とすると,$\mathscr{O}$に$\mathbb{R}^n-A$(これはopen)を合わせたものは$B$を覆う。系\ref{cor1.6}より$B$はcompactであるのでその中の有限個$U_1,U_2,\cdots,U_k,\mathbb{R}^n-A$がすでに$B$を覆う。したがって$U_1,U_2,\cdots,U_k$は$A$を覆う。
\end{proof}