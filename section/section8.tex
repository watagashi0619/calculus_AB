\part{積分}

\section{閉方体上での積分}
\begin{dfn*}[1次元の分割]
	$[a,b]$の分割$\Delta_0$とは
	\[
		\Delta_0=\{x_i\}_{i=0,\cdots,k}\ (a=x_0<x_1<\cdots<x_k=b)
	\]
\end{dfn*}
\begin{dfn*}[閉方体の分割]
閉方体$[a_1,b_1]\times \cdots \times [a_n,b_n]$の分割$\Delta$を各区間$[a_i,b_i]$の分割$\Delta_i$の組$\Delta:=(\Delta_1,\cdots,\Delta_n)$と定義する。
\end{dfn*}
\begin{dfn*}[小方体]
一般に,$\Delta_i$が$[a_i,b_i]$を$N_i$個に分割すれば,$\Delta=(\Delta_1,\cdots,\Delta_n)$は$[a_1,b_1]\times\cdots\times[a_n,b_n]$を$N=N_1N_2\cdots N_n$個の小閉方体に分割する。この小閉方体を小方体という。
\end{dfn*}
\begin{dfn*}[下限和・上限和]
$A$を小方体,$f:A\to\mathbb{R}$を有界関数とする。$A$の分割$\Delta$と$\Delta$の小方体$B$に対して
\[
	m_B(f):=\inf\{f(x):x\in B\}
\]
\[
	M_B(f):=\sup\{f(x):x\in B\}
\]
とおく。小方体$B$の体積を$|B|$とおく。閉方体$[a_1,b_1]\times\cdots\times[a_n,b_n]$の体積は$(b_1-a_1)(b_2-a_2)\cdots(b_n-a_n)$とする。ここで
\[
s(f,\Delta):=\sum_B m_B(f)|B|
\]
\[
S(f,\Delta):=\sum_B M_B(f)|B|
\]
と定義し,$s(f,\Delta)$を$f$の分割$\Delta$に関する下限和,$S(f,\Delta)$を$f$の分割$\Delta$に関する上限和という。このとき,明らかに
\[
s(f,\Delta)\leq S(f,\Delta)
\]
が成り立つ。
\end{dfn*}

\begin{framed}
	\begin{lem}\label{lem3.1}
		分割$\Delta'$が分割$\Delta$の細分\footnotemark ならば,$s(f,\Delta)\leq s(f,\Delta')$,$S(f,\Delta')\leq S(f,\Delta)$
	\end{lem}
\end{framed}
\footnotetext{$\Delta'$の小方体はすべて$\Delta$のある小方体の分割となる}

\begin{proof}
$\Delta$の各小方体$B$は何個かの$\Delta'$の小方体$B_1,\cdots,B_\alpha$に分割される。よって$|B|=|B_1|+|B_2|+\cdots+|B_\alpha|$。また,$B\supset B_i$だから$m_B(f)\leq M_B(f)$。したがって
\[
    \begin{split}
    m_B(f)\cdot |B|&=m_B(f)\cdot|B_1|+\cdots+m_B(f)\cdot|B_\alpha|\\
    &\leq m_{B_1}(f)\cdot|B_1|+\cdots+m_{B_\alpha}(f)\cdot|B_\alpha|\footnotemark
    \end{split}
\]
\footnotetext{
$\Delta'$の$B$上の下限和を表す。
}
すべての$B$に関する左辺の和が$s(f,\Delta)$,右辺の和が$s(f,\Delta')$だから,
\[
    s(f,\Delta)\leq s(f,\Delta')
\]
\end{proof}

\begin{framed}
    \begin{cor}
        任意の分割$\Delta$,$\Delta'$に対して$s(f,\Delta)\leq s(f,\Delta')$
    \end{cor}
\end{framed}

\begin{proof}
    $\Delta''$を$\Delta$と$\Delta'$の両方の細分であるような分割とする。このとき
    \[
        s(f,\Delta)\leq s(f,\Delta'')\leq S(f,\Delta'')\leq S(f,\Delta')
    \]
    以上より
    \[
        \sup_{\Delta} s(f,\Delta)\leq \inf_{\Delta} S(f,\Delta)
    \]
    が\textbf{常に}成り立つ。
\end{proof}
\begin{dfn*}[リーマン可積分 Riemann Integrable]
    \[
        \sup_{\Delta} s(f,\Delta)= \inf_{\Delta} S(f,\Delta)
    \]
    のとき,有理関数$f$は閉方体$A$上で可積分(Riemann Integrable)であるという。
\end{dfn*}

\begin{framed}
    \begin{thm}
        有界関数$f:A\to\mathbb{R}$が可積分であるための必要十分条件は
        \begin{equation}\tag{*}\forall\epsilon>0,\exists\Delta:Aの分割\ {\rm s.t. } S(f,\Delta)-s(f,\Delta)<\epsilon
        \end{equation}
    \end{thm}
\end{framed}

\begin{proof}\
    \par\noindent\textbf{((*)$\Rightarrow$$f$が$A$上可積分)}\\
    定義から$s(f,\Delta)\leq\sup_{\Delta}s(f,\Delta)$,$\inf_{\Delta}S(f,\Delta)\leq S(f,\Delta)$。仮定から$\forall\epsilon>0$に対して$\inf_{\Delta}S(f,\Delta)-\sup_{\Delta}s(f,\Delta)\leq S(f,\Delta)-s(f,\Delta) <\epsilon $が成り立つので
    \[
    	\sup_{\Delta} s(f,\Delta)= \inf_{\Delta} S(f,\Delta)
    \]
    となり,$f$は可積分。\\
    \textbf{($f$が$A$上可積分 $\Rightarrow$(*))}\\
    $f$が可積分ならば,
    \[
    	\forall\epsilon>0,\exists\Delta,\Delta':Aの分割\ {\rm s.t.}\ S(f,\Delta)-s(f,\Delta')<\epsilon
    \]
    $\Delta$と$\Delta'$の細分を$\Delta''$とすると,補題\ref{lem3.1}より
    \[
    	S(f,\Delta'')-s(f,\Delta'')\leq S(f,\Delta')-s(f,\Delta')<\epsilon
    \]
\end{proof}