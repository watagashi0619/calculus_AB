\part{多変数関数の微分}

\section{微分(全微分可能性)}

\paragraph{一変数の微分}

$f:\mathbb{R}\to\mathbb{R}$の微分
\[
\lim_{h\to 0}\frac{f(a+h)-f(a)}{h}=f'(a)
\]
(言い換え)$a\in\mathbb{R}$で微分可能とは,\\
\[
\exists\lambda:\mathbb{R}\to\mathbb{R}:線形写像\ {\rm s.t.}\ \lim_{h\to 0}\frac{f(a+h)-f(a)-\lambda(h)}{h}=0
\]
\footnote{線形写像なら$\lambda(h)=f'(a)h$。$y=px$みたいな。$\lambda(h)$と書いてるけど関数$\lambda$に$h$を代入しているわけではないことに注意。$\lambda h$と書いてもいい気がする。(行列$(\lambda_1,\lambda_2,\cdots,\lambda_m)$と${}^t(h_1,h_2,\cdots,h_m)$の積なので。)}

\begin{thm}[全微分]
関数$f:\mathbb{R}^n\to\mathbb{R}^m$が点$a\in\mathbb{R}^n$で微分可能(全微分可能)とは,
\[
\exists\lambda:\mathbb{R}^n\to\mathbb{R}^m:線形写像\ {\rm s.t.}\ \lim_{h\to 0}\frac{|f(a+h)-f(a)-\lambda(h)|}{|h|}=0
\]
分母は$\mathbb{R}^n$のnorm,分子は$\mathbb{R}^m$のnormであることに注意。$\lambda$は一次元での微分係数の一般化。この$\lambda$を$Df(a)$と書く。
\end{thm}

\newpage

\begin{framed}
	\begin{thm}
		$f:\mathbb{R}^n\to\mathbb{R}^m$が$a\in\mathbb{R}^n$で全微分可能のとき,$\mathbb{R}^n$から$\mathbb{R}^m$の線形写像$\lambda$で$\displaystyle \lim_{h\to 0}\frac{|f(a+h)-f(a)-\lambda(h)|}{|h|}=0 $を満たすものは1つしかない。
	\end{thm}
\end{framed}

\footnote{一意性の証明は2つ持ってきて矛盾させる。}

\begin{proof}
	線形写像$\mu:\mathbb{R}^n\to\mathbb{R}^m$も$\displaystyle \lim_{h\to 0}\frac{|f(a+h)-f(a)-\lambda(h)|}{|h|}=0$を満たすとする。
	\[
	\begin{split}
	\lim_{h\to 0}\frac{|\lambda(h)-\mu(h)|}{|h|}
	&= \lim_{h\to0}\frac{|\lambda(h)-\{f(a+h)-f(a)\}+\{f(a+h)-f(a)\}-\mu(h)|}{|h|}\\
	&\leq\lim_{h\to0}\frac{|f(a+h)-f(a)-\lambda(h)|}{|h|}+\lim_{h\to0}\frac{|f(a+h)-f(a)-\mu(h)|}{|h|}\footnotemark\\
	&= 0\footnotemark
	\end{split}
	\]
	\footnotetext{normの三角不等式。}
	\footnotetext{微分の定義と仮定から。}
	よって
	\[
	\lim_{h\to0}\frac{|\lambda(h)-\mu(h)|}{|h|}=0
	\]
	よって,$\forall x\in\mathbb{R}^n$に対し$t\to0$のとき$tx\to0$となるので,$\forall x\neq0$に対し上式より($h=tx$として)
	\[
	\begin{split}
	0 &=\lim_{t\to0}\frac{|\lambda(tx)-\mu(tx)|}{tx}\\
	&= \lim_{t\to0}\frac{|\lambda(x)-\mu(x)|}{|x|}\footnotemark\\
	&= \frac{|\lambda(x)-\mu(x)|}{|x|}
	\end{split}
	\]
	よって$\lambda(x)=\mu(x)$$(\forall x\in\mathbb{R}^n)$となる。
	\footnotetext{
	線形写像$\lambda(h)=Ah$は$\lambda(tx)=t\lambda(x),\lambda(0)=0$
	}
\end{proof}

\footnote{$\lambda:\mathbb{R}^n\to\mathbb{R}^m\Leftrightarrow A:m\times n$行列$\lambda(x)=Ax$}

\newpage

\paragraph{$Df(a):\mathbb{R}^n\to\mathbb{R}^m$について}
これは$\mathbb{R}^n$から$\mathbb{R}^m$への線形写像なので$\mathbb{R}^n$と$\mathbb{R}^m$の標準基底に関する表現行列($m\times n$行列)を用いると具体的に表せる。この$m\times n$行列を$f$の$a$でのヤコビ行列といい,$f'(a)$とかく。

$f:\mathbb{R}^n\to\mathbb{R}^m$,$x=(x^1,x^2,\cdots,x^n)$$f={}^{t}(f^1,f^2,\cdots,f^m)$
\[
Df(a)=f'(a)=
\begin{pmatrix}
u_{11} & u_{12} & \cdots & \cdots & \cdots & u_{1n}\\
u_{21} & u_{22} &        &        &        & u_{2n} \\
\vdots &        & \ddots &        &        & \vdots \\
\vdots &        &        & u_{ij} &        & \vdots \\
\vdots &        &        &        & \ddots & \vdots \\
u_{m1} & u_{m2} & \cdots & \cdots & \cdots & u_{mn}
\end{pmatrix}
\]
ただし$\displaystyle u_{ij}=\frac{\partial f^i}{\partial x^j}(a)$である。

\begin{note}\
	\begin{itemize}
		\item 関数$f$が$\mathbb{R}^n$の点$a$を含むある開集合上だけで定義されている場合でも$Df(a)$は定義できる。($Df(a):\mathbb{R}^n\to\mathbb{R}^m$:linear)
		\item 関数$f:A\to\mathbb{R}^m$が$A$だけでしか定義されていない場合は$f$が$A$を含むある開集合上の可微分関数に拡張できる時,$f$は$A$上微分可能という。
		\item 全微分可能ならば連続である。
	\end{itemize}
\end{note}

\begin{example}
$f(x,y)={}^{t}(f^1(x,y),f^2(x,y))={}^{t}(xy,x+y)$\\
これを	$(x,y)=(a,b)$で微分$(a,b)\to(a+h,b+k)$
\[
\begin{split}
\lim_{
\tiny
\begin{pmatrix}
h \\
k \\
\end{pmatrix}
\to
\begin{pmatrix}
0 \\
0 \\
\end{pmatrix}}
\frac{
\left|
\begin{pmatrix}
(a+h)(b+k) \\
(a+h)+(b+k) \\
\end{pmatrix}
-
\begin{pmatrix}
ab \\
a+b \\
\end{pmatrix}
-
\begin{pmatrix}
b & a \\
1 & 1\\
\end{pmatrix}
\begin{pmatrix}
h \\
k \\
\end{pmatrix}
\right|
}
{\left|\begin{pmatrix}
h \\
k \\
\end{pmatrix}\right|}
&=
\lim_{
\tiny
\begin{pmatrix}
h \\
k \\
\end{pmatrix}
\to
\begin{pmatrix}
0 \\
0 \\
\end{pmatrix}}
\frac{\left|
\begin{pmatrix}
hk \\
0 \\
\end{pmatrix}
\right|}{\left|
\begin{pmatrix}
h \\
k \\
\end{pmatrix}
\right|}
\\
&=
\lim_{
\tiny
\begin{pmatrix}
h \\
k \\
\end{pmatrix}
\to
\begin{pmatrix}
0 \\
0 \\
\end{pmatrix}}
\frac{|hk|}{\sqrt{h^2+k^2}}\to0
\end{split}
\]
\end{example}
\footnote{
$
\begin{pmatrix}
ak+bh+hk \\
h+k \\
\end{pmatrix}
=
\begin{pmatrix}
b & a \\
1 & 1\\
\end{pmatrix}
\begin{pmatrix}
h \\
k \\
\end{pmatrix}
+
\begin{pmatrix}
h,kの\\
二次以上 \\
\end{pmatrix}
$
}
